
\documentclass[12pt]{article}
\usepackage[utf8]{inputenc}
\usepackage{amsmath}
\usepackage{color,soul}
\usepackage{fancyhdr}
\pagestyle{fancy}

% Default fixed font does not support bold face
\DeclareFixedFont{\ttb}{T1}{txtt}{bx}{n}{12} % for bold
\DeclareFixedFont{\ttm}{T1}{txtt}{m}{n}{12}  % for normal

% Custom colors
\usepackage{color}
\definecolor{deepblue}{rgb}{0,0,0.5}
\definecolor{deepred}{rgb}{0.6,0,0}
\definecolor{deepgreen}{rgb}{0,0.5,0}

\usepackage{listings}

% Python style for highlighting
\newcommand\pythonstyle{\lstset{
		language=Python,
		basicstyle=\ttm,
		otherkeywords={self},             % Add keywords here
		keywordstyle=\ttb\color{deepblue},
		emph={MyClass,__init__},          % Custom highlighting
		emphstyle=\ttb\color{deepred},    % Custom highlighting style
		stringstyle=\color{deepgreen},
		frame=tb,                         % Any extra options here
		showstringspaces=false            % 
}}


% Python environment
\lstnewenvironment{python}[1][]
{
	\pythonstyle
	\lstset{#1}
}
{}

% Python for external files
\newcommand\pythonexternal[2][]{{
		\pythonstyle
		\lstinputlisting[#1]{#2}}}

% Python for inline
\newcommand\pythoninline[1]{{\pythonstyle\lstinline!#1!}}

\usepackage[utf8]{inputenc}
\usepackage[english]{babel}
\usepackage{amssymb}
\usepackage{amsfonts}
\usepackage{multicol}

\newenvironment{lemma}{\paragraph{Lemma:\hfill}}{}
\newenvironment{proof}{\paragraph{Proof:\hfill}}{\hfill$\square$}
\newenvironment{proofMathInduct}{\paragraph{Proof by Mathematical Induction:\hfill}}{\hfill$\square$}
\newenvironment{proofMathInductStrong}{\paragraph{Proof by Mathematical Induction (Strong Form):\hfill}}{\hfill$\square$}
\newenvironment{proofContradiction}{\paragraph{Proof by contradiction: \hfill}}{\hfill$\square$}

\newenvironment{theorem}{\paragraph{Theorem:\hfill}}{\hfill}
\newenvironment{definition}{
	\paragraph{Definition: } 
	\hfill
	\\
	\indent  
}{\hfill}

\title{Linear Algebra Final Study Guide}
\author{Andrew Reed}
\date{\today}

\begin{document}
\maketitle

\section{The Definitions}

\begin{definition}
	A vector $\textbf{v}$ is a \textbf{linear combination} of $\textbf{v}_1, \textbf{v}_2, \dots, \textbf{v}_k$ if there are scalars, $c_1, c_2, \dots, c_k$ such that $\textbf{v} = c_1 \times \textbf{v}_1 + c_2 \times \textbf{v}_2 + \dots + c_k + \textbf{v}_k$.
\end{definition}

\begin{definition}
	If $S = {\textbf{v}_1, \textbf{v}_2, \dots, \textbf{v}_k}$, is a set of vectors in $\mathcal{R}^n$, then the set of all linear combinations of $\textbf{v}_1, \textbf{v}_2, \dots, \textbf{v}_k$ is called a \textbf{span} of $\textbf{v}_1, \textbf{v}_2, \dots, \textbf{v}_k$ and denoted by $span(\textbf{v}_1, \textbf{v}_2, \dots, \textbf{v}_k)$ or $span(S) $.
\end{definition}

\begin{definition}
	A set of vectors $\textbf{v}_1, \textbf{v}_2, \dots, \textbf{v}_k$ is \textbf{linear dependent} if the scalars $c_1, c_2, \dots, c_k$, at least one of which is not zero, such that
	\[
	c_1 \times \textbf{v}_1 + c_2 \times \textbf{v}_2 + \dots + c_k \times \textbf{v}_k = 0
	\]
\end{definition}

\begin{definition}
	A \textbf{basis} for a subspace $S$ of $\mathcal{R}^n$ is a set of vectors in $S$ that
	\begin{enumerate}
		\item spans $S$
		\item Is linear independent.
	\end{enumerate}
\end{definition}



\newpage


\section{Basic Vector Mathematics}

\paragraph{Finding vectors from points}
\hfill
\\
\indent Given two points $P(a, b)$ and $Q(c, d)$ then a vector $\textbf{v}$ where $\textbf{v} = PQ$, then; 
\[
\textbf{v} = 
\begin{bmatrix}
	c - a \\
	d - b \\
\end{bmatrix}
\]

\paragraph{Unit Vectors}
\hfill
\\
\indent Given a vector \textbf{v} the unit vector for said vector is
\[
	\left( \frac{1}{||\textbf{v}||} \right)  \textbf{v}
\]

\paragraph{Distance}
\hfill
\\
\indent The distance between two vectors is denoted $d(\textbf{u}, \textbf{v})$ and defined as
\[
	d(\textbf{u}, \textbf{v})) = || \textbf{u} - \textbf{v} ||
\]



\section{Eigenvalues, Eigenvectors and Eigenspaces}

\section{Standard Matrix}

\newpage

\section{Spans}

\paragraph{Spans of Matrices}
Find $span(A_1, A_2)$ given 
\[
a_1 = 
\begin{bmatrix}
1 & 2 \\
-1 & 4 \\
\end{bmatrix}
,
A_2 = 
\begin{bmatrix}
0 & 1 \\
3 & 4 \\
\end{bmatrix}
\]

\underline{Step 1}: Establish an augmented matrix.
\[
\begin{bmatrix}
	1  & 0 & | &a\\
	2  & 1 & | & b\\
	-1  & 3 & | & c\\
	4  & 4 & | & d\\
\end{bmatrix}
\]

\underline{Step 2}: Row reduce to row echelon form

\[
\begin{bmatrix}
1  & 0 & | &a\\
2  & 1 & | & b\\
-1  & 3 & | & c\\
4  & 4 & | & d\\
\end{bmatrix}
\rightarrow
\begin{bmatrix}
1  & 0 & | &a\\
0  & 1 & | & b - 2a\\
0  & 0 & | & 7a - 3b + c\\
0  & 0 & | & 4a - 4b + d\\
\end{bmatrix}
\]

\underline{Step 3}: Solve for 0 in rows which contain all 0's
\[
7a - 3b + c = 0
\]
\[
c = -7a + 3b
\]
\hfill
\\
\[
4a - 4b + d = 0
\]
\[
d = -4a - 4b
\]

\underline{Step 4}: use the values obtained to create a matrix
\[
\begin{bmatrix}
	a & b-2a \\
	-7a + 3b & -4a + 4b \\
\end{bmatrix}
\]
\end{document}